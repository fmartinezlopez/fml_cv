\datedsubsection{2021 -- present}
	{%
		Queen Mary University of London, UK \newline \hfill and
		University of Southampton, UK}
	{%
		\textbf{Ph.D.}~in Particle Physics}
	{%
		\begin{justify}\noindent
			I have been working within the DUNE collaboration under the supervision of Dr. Linda Cremonesi, Prof. Stefano Moretti and Prof. Claire Shepherd-Themistocleous. I expect to complete my PhD by the end of 2024. \newline
			My research focuses on the simulation and reconstruction of the gaseous argon near detector that DUNE will deploy in Phase II. I developed the particle identification algorithms of the detector using traditional reconstruction techniques as well as machine learning methods. Moreover, I put together the infrastructure to produce analysis-ready files for the detector, to be used in the long baseline analysis. Also, I explored the capabilities of the far detector to detect neutrino fluxes from Dark Matter annihilations inside the Sun. \newline
			Furthermore, I have worked on the data acquisition system of the DUNE far detector, developing techniques to allow for triggering on low-energy events.
			\newline \\
			From May 2021 to June 2022 I was based at Rutherford Appleton Laboratory (Didcot, Oxfordshire, UK), where I primarily worked on the validation of the firmware-based trigger primitive generation for the DUNE FD. \newline
			From June 2022 to July 2023 I was based at CERN (Geneva, Switzerland), working on the commissioning of the DAQ for ProtoDUNE-II at the HD and VD ColdBox setups.
		\end{justify}
	}

\datedsubsection{2019 -- 2020}
		{%
			Universitat de València, Spain}
		{%
			\textbf{M.Sc.}~in Theoretical Physics}
		{%
			\begin{justify}\noindent
				Dissertation title: ``TeV-scale bulk neutrino in warped extra-dimensions as a DM candidate''. \newline
				Study of the possibility to identify one of the right-handed neutrinos entering the type-I seesaw mechanism as a Dark Matter candidate, within the warped extra-dimensions paradigm. I explored the parameter space of the model, confront it to the current experimental bounds, determine its potential flaws and finally propose a series of extensions to solve them.
			\end{justify}
		}

\datedsubsection{2015 -- 2019}
			{%
				Universidad de Murcia, Spain}
			{%
				\textbf{B.Sc.}~in Physics}
			{%
				\begin{justify}\noindent
					Dissertation title: ``The information loss paradox''. \newline
					Review of the black hole information paradox, discussing the case of a free scalar field propagating in a curved spacetime. I studied how the paradox arises from the entanglement between the radiation quanta and the mass in the black hole, and showed how asymptotic symmetries might help solve the problem.
				\end{justify}
			}